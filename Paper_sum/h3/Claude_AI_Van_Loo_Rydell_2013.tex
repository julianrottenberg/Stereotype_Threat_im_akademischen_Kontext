% Options for packages loaded elsewhere
\PassOptionsToPackage{unicode}{hyperref}
\PassOptionsToPackage{hyphens}{url}
%
\documentclass[
  doc, a4paper]{apa7}
\usepackage{amsmath,amssymb}
\usepackage{iftex}
\ifPDFTeX
  \usepackage[T1]{fontenc}
  \usepackage[utf8]{inputenc}
  \usepackage{textcomp} % provide euro and other symbols
\else % if luatex or xetex
  \usepackage{unicode-math} % this also loads fontspec
  \defaultfontfeatures{Scale=MatchLowercase}
  \defaultfontfeatures[\rmfamily]{Ligatures=TeX,Scale=1}
\fi
\usepackage{lmodern}
\ifPDFTeX\else
  % xetex/luatex font selection
\fi
% Use upquote if available, for straight quotes in verbatim environments
\IfFileExists{upquote.sty}{\usepackage{upquote}}{}
\IfFileExists{microtype.sty}{% use microtype if available
  \usepackage[]{microtype}
  \UseMicrotypeSet[protrusion]{basicmath} % disable protrusion for tt fonts
}{}
\makeatletter
\@ifundefined{KOMAClassName}{% if non-KOMA class
  \IfFileExists{parskip.sty}{%
    \usepackage{parskip}
  }{% else
    \setlength{\parindent}{0pt}
    \setlength{\parskip}{6pt plus 2pt minus 1pt}}
}{% if KOMA class
  \KOMAoptions{parskip=half}}
\makeatother
\usepackage{xcolor}
\usepackage{graphicx}
\makeatletter
\def\maxwidth{\ifdim\Gin@nat@width>\linewidth\linewidth\else\Gin@nat@width\fi}
\def\maxheight{\ifdim\Gin@nat@height>\textheight\textheight\else\Gin@nat@height\fi}
\makeatother
% Scale images if necessary, so that they will not overflow the page
% margins by default, and it is still possible to overwrite the defaults
% using explicit options in \includegraphics[width, height, ...]{}
\setkeys{Gin}{width=\maxwidth,height=\maxheight,keepaspectratio}
% Set default figure placement to htbp
\makeatletter
\def\fps@figure{htbp}
\makeatother
\setlength{\emergencystretch}{3em} % prevent overfull lines
\providecommand{\tightlist}{%
  \setlength{\itemsep}{0pt}\setlength{\parskip}{0pt}}
\setcounter{secnumdepth}{-\maxdimen} % remove section numbering
% Make \paragraph and \subparagraph free-standing
\ifx\paragraph\undefined\else
  \let\oldparagraph\paragraph
  \renewcommand{\paragraph}[1]{\oldparagraph{#1}\mbox{}}
\fi
\ifx\subparagraph\undefined\else
  \let\oldsubparagraph\subparagraph
  \renewcommand{\subparagraph}[1]{\oldsubparagraph{#1}\mbox{}}
\fi
% definitions for citeproc citations
\NewDocumentCommand\citeproctext{}{}
\NewDocumentCommand\citeproc{mm}{%
  \begingroup\def\citeproctext{#2}\cite{#1}\endgroup}
\makeatletter
 % allow citations to break across lines
 \let\@cite@ofmt\@firstofone
 % avoid brackets around text for \cite:
 \def\@biblabel#1{}
 \def\@cite#1#2{{#1\if@tempswa , #2\fi}}
\makeatother
\newlength{\cslhangindent}
\setlength{\cslhangindent}{1.5em}
\newlength{\csllabelwidth}
\setlength{\csllabelwidth}{3em}
\newenvironment{CSLReferences}[2] % #1 hanging-indent, #2 entry-spacing
 {\begin{list}{}{%
  \setlength{\itemindent}{0pt}
  \setlength{\leftmargin}{0pt}
  \setlength{\parsep}{0pt}
  % turn on hanging indent if param 1 is 1
  \ifodd #1
   \setlength{\leftmargin}{\cslhangindent}
   \setlength{\itemindent}{-1\cslhangindent}
  \fi
  % set entry spacing
  \setlength{\itemsep}{#2\baselineskip}}}
 {\end{list}}
\usepackage{calc}
\newcommand{\CSLBlock}[1]{\hfill\break\parbox[t]{\linewidth}{\strut\ignorespaces#1\strut}}
\newcommand{\CSLLeftMargin}[1]{\parbox[t]{\csllabelwidth}{\strut#1\strut}}
\newcommand{\CSLRightInline}[1]{\parbox[t]{\linewidth - \csllabelwidth}{\strut#1\strut}}
\newcommand{\CSLIndent}[1]{\hspace{\cslhangindent}#1}
\ifLuaTeX
\usepackage[bidi=basic]{babel}
\else
\usepackage[bidi=default]{babel}
\fi
\babelprovide[main,import]{english}
% get rid of language-specific shorthands (see #6817):
\let\LanguageShortHands\languageshorthands
\def\languageshorthands#1{}
% Manuscript styling
\usepackage{upgreek}
\captionsetup{font=singlespacing,justification=justified}

% Table formatting
\usepackage{longtable}
\usepackage{lscape}
% \usepackage[counterclockwise]{rotating}   % Landscape page setup for large tables
\usepackage{multirow}		% Table styling
\usepackage{tabularx}		% Control Column width
\usepackage[flushleft]{threeparttable}	% Allows for three part tables with a specified notes section
\usepackage{threeparttablex}            % Lets threeparttable work with longtable

% Create new environments so endfloat can handle them
% \newenvironment{ltable}
%   {\begin{landscape}\centering\begin{threeparttable}}
%   {\end{threeparttable}\end{landscape}}
\newenvironment{lltable}{\begin{landscape}\centering\begin{ThreePartTable}}{\end{ThreePartTable}\end{landscape}}

% Enables adjusting longtable caption width to table width
% Solution found at http://golatex.de/longtable-mit-caption-so-breit-wie-die-tabelle-t15767.html
\makeatletter
\newcommand\LastLTentrywidth{1em}
\newlength\longtablewidth
\setlength{\longtablewidth}{1in}
\newcommand{\getlongtablewidth}{\begingroup \ifcsname LT@\roman{LT@tables}\endcsname \global\longtablewidth=0pt \renewcommand{\LT@entry}[2]{\global\advance\longtablewidth by ##2\relax\gdef\LastLTentrywidth{##2}}\@nameuse{LT@\roman{LT@tables}} \fi \endgroup}

% \setlength{\parindent}{0.5in}
% \setlength{\parskip}{0pt plus 0pt minus 0pt}

% Overwrite redefinition of paragraph and subparagraph by the default LaTeX template
% See https://github.com/crsh/papaja/issues/292
\makeatletter
\renewcommand{\paragraph}{\@startsection{paragraph}{4}{\parindent}%
  {0\baselineskip \@plus 0.2ex \@minus 0.2ex}%
  {-1em}%
  {\normalfont\normalsize\bfseries\itshape\typesectitle}}

\renewcommand{\subparagraph}[1]{\@startsection{subparagraph}{5}{1em}%
  {0\baselineskip \@plus 0.2ex \@minus 0.2ex}%
  {-\z@\relax}%
  {\normalfont\normalsize\itshape\hspace{\parindent}{#1}\textit{\addperi}}{\relax}}
\makeatother

\makeatletter
\usepackage{etoolbox}
\patchcmd{\maketitle}
  {\section{\normalfont\normalsize\abstractname}}
  {\section*{\normalfont\normalsize\abstractname}}
  {}{\typeout{Failed to patch abstract.}}
\patchcmd{\maketitle}
  {\section{\protect\normalfont{\@title}}}
  {\section*{\protect\normalfont{\@title}}}
  {}{\typeout{Failed to patch title.}}
\makeatother

\usepackage{xpatch}
\makeatletter
\xapptocmd\appendix
  {\xapptocmd\section
    {\addcontentsline{toc}{section}{\appendixname\ifoneappendix\else~\theappendix\fi\\: #1}}
    {}{\InnerPatchFailed}%
  }
{}{\PatchFailed}
\keywords{keywords\newline\indent Word count: X}
\usepackage{csquotes}
\makeatletter
\renewcommand{\paragraph}{\@startsection{paragraph}{4}{\parindent}%
  {0\baselineskip \@plus 0.2ex \@minus 0.2ex}%
  {-1em}%
  {\normalfont\normalsize\bfseries\typesectitle}}

\renewcommand{\subparagraph}[1]{\@startsection{subparagraph}{5}{1em}%
  {0\baselineskip \@plus 0.2ex \@minus 0.2ex}%
  {-\z@\relax}%
  {\normalfont\normalsize\bfseries\itshape\hspace{\parindent}{#1}\textit{\addperi}}{\relax}}
\makeatother

\ifLuaTeX
  \usepackage{selnolig}  % disable illegal ligatures
\fi
\usepackage{bookmark}
\IfFileExists{xurl.sty}{\usepackage{xurl}}{} % add URL line breaks if available
\urlstyle{same}
\hypersetup{
  pdftitle={Van Loo and Rydell (2013)},
  pdflang={en-EN},
  pdfkeywords={keywords},
  hidelinks,
  pdfcreator={LaTeX via pandoc}}

\title{Van Loo and Rydell (2013)}
\author{\phantom{0}}
\date{}


\shorttitle{Van Loo and Rydell (2013)}

\affiliation{\phantom{0}}

\begin{document}
\maketitle

\subsubsection{If the study has a broad focus and this data extraction focuses on just one component of the study, please specify this here}\label{if-the-study-has-a-broad-focus-and-this-data-extraction-focuses-on-just-one-component-of-the-study-please-specify-this-here}

\begin{itemize}
\tightlist
\item[$\boxtimes$]
  Not applicable (whole study is focus of data extraction)
\end{itemize}

\subsection{Study aim(s) and rationale}\label{study-aims-and-rationale}

\subsubsection{Was the study informed by, or linked to, an existing body of empirical and/or theoretical research?}\label{was-the-study-informed-by-or-linked-to-an-existing-body-of-empirical-andor-theoretical-research}

\begin{itemize}
\tightlist
\item[$\boxtimes$]
  Explicitly stated (please specify)
\end{itemize}

The study was informed by existing research on stereotype threat, which shows that awareness of negative stereotypes about one's group can lead to reduced performance. It was also linked to research on power and its effects on cognitive resources.

\subsubsection{Do authors report how the study was funded?}\label{do-authors-report-how-the-study-was-funded}

\begin{itemize}
\tightlist
\item[$\boxtimes$]
  Explicitly stated (please specify)
\end{itemize}

The research was funded in part by a National Science Foundation Graduate Research Fellowship to the first author.

\subsection{Study research question(s) and its policy or practice focus}\label{study-research-questions-and-its-policy-or-practice-focus}

\subsubsection{What is/are the topic focus/foci of the study?}\label{what-isare-the-topic-focusfoci-of-the-study}

The study focuses on how feelings of power can protect women from the negative effects of stereotype threat on math performance, and examines working memory capacity as a potential mechanism.

\subsubsection{What is/are the population focus/foci of the study?}\label{what-isare-the-population-focusfoci-of-the-study}

The population focus is undergraduate women.

\subsubsection{What is the relevant age group?}\label{what-is-the-relevant-age-group}

\begin{itemize}
\tightlist
\item[$\boxtimes$]
  17 - 20
\end{itemize}

\subsubsection{What is the sex of the population focus/foci?}\label{what-is-the-sex-of-the-population-focusfoci}

\begin{itemize}
\tightlist
\item[$\boxtimes$]
  Female only
\end{itemize}

\subsubsection{What is/are the educational setting(s) of the study?}\label{what-isare-the-educational-settings-of-the-study}

\begin{itemize}
\tightlist
\item[$\boxtimes$]
  Higher education institution
\end{itemize}

\subsubsection{In Which country or cuntries was the study carried out?}\label{in-which-country-or-cuntries-was-the-study-carried-out}

\begin{itemize}
\tightlist
\item[$\boxtimes$]
  Explicitly stated (please specify)
\end{itemize}

The study was carried out in the United States.

\subsubsection{Please describe in more detail the specific phenomena, factors, services, or interventions with which the study is concerned}\label{please-describe-in-more-detail-the-specific-phenomena-factors-services-or-interventions-with-which-the-study-is-concerned}

The study examines how feelings of power interact with stereotype threat to affect women's math performance and working memory capacity. It investigates whether feeling powerful can protect women from the negative effects of stereotype threat on math performance by preserving working memory capacity.

\subsubsection{What are the study reserach questions and/or hypotheses?}\label{what-are-the-study-reserach-questions-andor-hypotheses}

\begin{itemize}
\tightlist
\item[$\boxtimes$]
  Explicitly stated (please specify)
\end{itemize}

The authors hypothesized that women made to feel powerful would be protected from working memory decrements in response to stereotype threat, and thus would not show strong stereotype threat-based performance effects. In contrast, women in low power and control conditions were expected to show decreased math performance when given stereotype threat instructions, accounted for by reductions in working memory capacity.

\subsection{Methods - Design}\label{methods---design}

\subsubsection{Which variables or concepts, if any, does the study aim to measure or examine?}\label{which-variables-or-concepts-if-any-does-the-study-aim-to-measure-or-examine}

\begin{itemize}
\tightlist
\item[$\boxtimes$]
  Explicitly stated (please specify)
\end{itemize}

The study measures power (manipulated as high, low, or control), stereotype threat (threat or no threat instructions), working memory capacity, math performance, and threat-based concern.

\subsubsection{Study timing}\label{study-timing}

\begin{itemize}
\tightlist
\item[$\boxtimes$]
  Cross-sectional
\end{itemize}

\subsubsection{If the study is an evaluation, when were measurements of the variable(s) used for outcome made, in relation to the intervention?}\label{if-the-study-is-an-evaluation-when-were-measurements-of-the-variables-used-for-outcome-made-in-relation-to-the-intervention}

\begin{itemize}
\tightlist
\item[$\boxtimes$]
  Only after
\end{itemize}

Measurements of working memory capacity and math performance were made after the power manipulation and stereotype threat instructions.

\subsection{Methods - Groups}\label{methods---groups}

\subsubsection{If comparisons are being made between two or more groups, please specify the basis of any divisions made for making these comparisons.}\label{if-comparisons-are-being-made-between-two-or-more-groups-please-specify-the-basis-of-any-divisions-made-for-making-these-comparisons.}

\begin{itemize}
\tightlist
\item[$\boxtimes$]
  Prospecitive allocation into more than one group (e.g.~allocation to different interventions, or allocation to intervention and control groups)
\end{itemize}

Participants were randomly assigned to power conditions (high, low, control) and stereotype threat conditions (threat, no threat).

\subsubsection{How do the groups differ?}\label{how-do-the-groups-differ}

\begin{itemize}
\tightlist
\item[$\boxtimes$]
  Explicityly stated (please specify)
\end{itemize}

The groups differ in their assigned power condition (high, low, or control) and whether they received stereotype threat or no threat instructions.

\subsubsection{Number of groups}\label{number-of-groups}

\begin{itemize}
\tightlist
\item[$\boxtimes$]
  Six
\end{itemize}

There were six groups: high power-threat, high power-no threat, low power-threat, low power-no threat, control power-threat, and control power-no threat.

\subsubsection{Was the assignment of participants to interventions randomised?}\label{was-the-assignment-of-participants-to-interventions-randomised}

\begin{itemize}
\tightlist
\item[$\boxtimes$]
  Random
\end{itemize}

\subsubsection{Where there was prospective allocation to more than one group, was the allocation sequence concealed from participants and those enrolling them until after enrolment?}\label{where-there-was-prospective-allocation-to-more-than-one-group-was-the-allocation-sequence-concealed-from-participants-and-those-enrolling-them-until-after-enrolment}

\begin{itemize}
\tightlist
\item[$\boxtimes$]
  Not stated/unclear (please specify)
\end{itemize}

The paper does not explicitly state whether allocation was concealed.

\subsubsection{Apart from the experimental intervention, did each study group receive the same level of care (that is, were they treated equally)?}\label{apart-from-the-experimental-intervention-did-each-study-group-receive-the-same-level-of-care-that-is-were-they-treated-equally}

\begin{itemize}
\tightlist
\item[$\boxtimes$]
  Yes
\end{itemize}

\subsubsection{Study design summary}\label{study-design-summary}

This was a 3 (power: low, control, high) x 2 (stereotype threat instructions: no threat, stereotype threat) between-subjects experimental design. Participants were randomly assigned to conditions. The study measured working memory capacity and math performance after manipulations of power and stereotype threat.

\subsection{Methods - Sampling strategy}\label{methods---sampling-strategy}

\subsubsection{Are the authors trying to produce findings that are representative of a given population?}\label{are-the-authors-trying-to-produce-findings-that-are-representative-of-a-given-population}

\begin{itemize}
\tightlist
\item[$\boxtimes$]
  Not stated/unclear (please specify)
\end{itemize}

The authors do not explicitly state that they are trying to produce representative findings.

\subsubsection{Which methods does the study use to identify people or groups of people to sample from and what is the sampling frame?}\label{which-methods-does-the-study-use-to-identify-people-or-groups-of-people-to-sample-from-and-what-is-the-sampling-frame}

\begin{itemize}
\tightlist
\item[$\boxtimes$]
  Not stated/unclear (please specify)
\end{itemize}

The sampling method is not explicitly stated, but participants appear to be undergraduate students.

\subsubsection{Which methods does the study use to select people or groups of people (from the sampling frame)?}\label{which-methods-does-the-study-use-to-select-people-or-groups-of-people-from-the-sampling-frame}

\begin{itemize}
\tightlist
\item[$\boxtimes$]
  Not stated/unclear (please specify)
\end{itemize}

The specific selection method is not stated.

\subsubsection{Planned sample size}\label{planned-sample-size}

\begin{itemize}
\tightlist
\item[$\boxtimes$]
  Not stated/unclear (please specify)
\end{itemize}

The planned sample size is not reported.

\subsection{Methods - Recruitment and consent}\label{methods---recruitment-and-consent}

\subsubsection{Which methods are used to recruit people into the study?}\label{which-methods-are-used-to-recruit-people-into-the-study}

\begin{itemize}
\tightlist
\item[$\boxtimes$]
  Not stated/unclear (please specify)
\end{itemize}

The recruitment methods are not explicitly stated.

\subsubsection{Were any incentives provided to recruit people into the study?}\label{were-any-incentives-provided-to-recruit-people-into-the-study}

\begin{itemize}
\tightlist
\item[$\boxtimes$]
  Explicitly stated (please specify)
\end{itemize}

Participants received partial course credit for participating.

\subsubsection{Was consent sought?}\label{was-consent-sought}

\begin{itemize}
\tightlist
\item[$\boxtimes$]
  Not stated/unclear (please specify)
\end{itemize}

The paper does not explicitly mention obtaining consent.

\subsubsection{Are there any other details relevant to recruitment and consent?}\label{are-there-any-other-details-relevant-to-recruitment-and-consent}

\begin{itemize}
\tightlist
\item[$\boxtimes$]
  No
\end{itemize}

\subsection{Methods - Actual sample}\label{methods---actual-sample}

\subsubsection{What was the total number of participants in the study (the actual sample)?}\label{what-was-the-total-number-of-participants-in-the-study-the-actual-sample}

\begin{itemize}
\tightlist
\item[$\boxtimes$]
  Explicitly stated (please specify)
\end{itemize}

The final sample was 131 women.

\subsubsection{What is the proportion of those selected for the study who actually participated in the study?}\label{what-is-the-proportion-of-those-selected-for-the-study-who-actually-participated-in-the-study}

\begin{itemize}
\tightlist
\item[$\boxtimes$]
  Not stated/unclear (please specify)
\end{itemize}

This information is not provided.

\subsubsection{Which country/countries are the individuals in the actual sample from?}\label{which-countrycountries-are-the-individuals-in-the-actual-sample-from}

\begin{itemize}
\tightlist
\item[$\boxtimes$]
  Explicitly stated (please specify)
\end{itemize}

The participants were from the United States.

\subsubsection{What ages are covered by the actual sample?}\label{what-ages-are-covered-by-the-actual-sample}

\begin{itemize}
\tightlist
\item[$\boxtimes$]
  Not stated/unclear (please specify)
\end{itemize}

The exact ages are not specified, but participants were undergraduate students.

\subsubsection{What is the socio-economic status of the individuals within the actual sample?}\label{what-is-the-socio-economic-status-of-the-individuals-within-the-actual-sample}

\begin{itemize}
\tightlist
\item[$\boxtimes$]
  Not stated/unclear (please specify)
\end{itemize}

Socio-economic status is not reported.

\subsubsection{What is the ethnicity of the individuals within the actual sample?}\label{what-is-the-ethnicity-of-the-individuals-within-the-actual-sample}

\begin{itemize}
\tightlist
\item[$\boxtimes$]
  Not stated/unclear (please specify)
\end{itemize}

Ethnicity information is not provided.

\subsubsection{What is known about the special educational needs of individuals within the actual sample?}\label{what-is-known-about-the-special-educational-needs-of-individuals-within-the-actual-sample}

\begin{itemize}
\tightlist
\item[$\boxtimes$]
  Not stated/unclear (please specify)
\end{itemize}

No information is provided about special educational needs.

\subsubsection{Is there any other useful information about the study participants?}\label{is-there-any-other-useful-information-about-the-study-participants}

\begin{itemize}
\tightlist
\item[$\boxtimes$]
  Not stated/unclear (please specify)
\end{itemize}

No other relevant information about the participants is provided.

\subsubsection{How representative was the achieved sample (as recruited at the start of the study) in relation to the aims of the sampling frame?}\label{how-representative-was-the-achieved-sample-as-recruited-at-the-start-of-the-study-in-relation-to-the-aims-of-the-sampling-frame}

\begin{itemize}
\tightlist
\item[$\boxtimes$]
  Not applicable (no sampling frame)
\end{itemize}

\subsubsection{If the study involves studying samples prospectively over time, what proportion of the sample dropped out over the course of the study?}\label{if-the-study-involves-studying-samples-prospectively-over-time-what-proportion-of-the-sample-dropped-out-over-the-course-of-the-study}

\begin{itemize}
\tightlist
\item[$\boxtimes$]
  Not applicable (not following samples prospectively over time)
\end{itemize}

\subsubsection{For studies that involve following samples prospectively over time, do the authors provide any information on whether and/or how those who dropped out of the study differ from those who remained in the study?}\label{for-studies-that-involve-following-samples-prospectively-over-time-do-the-authors-provide-any-information-on-whether-andor-how-those-who-dropped-out-of-the-study-differ-from-those-who-remained-in-the-study}

\begin{itemize}
\tightlist
\item[$\boxtimes$]
  Not applicable (not following samples prospectively over time)
\end{itemize}

\subsubsection{If the study involves following samples prospectively over time, do authors provide baseline values of key variables such as those being used as outcomes and relevant socio-demographic variables?}\label{if-the-study-involves-following-samples-prospectively-over-time-do-authors-provide-baseline-values-of-key-variables-such-as-those-being-used-as-outcomes-and-relevant-socio-demographic-variables}

\begin{itemize}
\tightlist
\item[$\boxtimes$]
  Not applicable (not following samples prospectively over time)
\end{itemize}

\subsection{Methods - Data collection}\label{methods---data-collection}

\subsubsection{Please describe the main types of data collected and specify if they were used (a) to define the sample; (b) to measure aspects of the sample as findings of the study?}\label{please-describe-the-main-types-of-data-collected-and-specify-if-they-were-used-a-to-define-the-sample-b-to-measure-aspects-of-the-sample-as-findings-of-the-study}

\begin{itemize}
\tightlist
\item[$\boxtimes$]
  Details
\end{itemize}

Data collected included: power manipulation check, mood measure, working memory capacity task performance, math task performance, and threat-based concern ratings. These were used to measure aspects of the sample as findings of the study.

\subsubsection{Which methods were used to collect the data?}\label{which-methods-were-used-to-collect-the-data}

\begin{itemize}
\tightlist
\item[$\boxtimes$]
  Self-completion questionnaire
\item[$\boxtimes$]
  Psychological test
\end{itemize}

\subsubsection{Details of data collection methods or tool(s).}\label{details-of-data-collection-methods-or-tools.}

\begin{itemize}
\tightlist
\item[$\boxtimes$]
  Explicitly stated (please specify)
\end{itemize}

Power was manipulated using an essay-writing task. Working memory was measured using a letter-memory task. Math performance was assessed using modular arithmetic problems. Threat-based concern was measured using a 3-item scale. Mood was measured using the PANAS.

\subsubsection{Who collected the data?}\label{who-collected-the-data}

\begin{itemize}
\tightlist
\item[$\boxtimes$]
  Not stated/unclear
\end{itemize}

\subsubsection{Do the authors describe any ways they addressed the reliability of their data collection tools/methods?}\label{do-the-authors-describe-any-ways-they-addressed-the-reliability-of-their-data-collection-toolsmethods}

\begin{itemize}
\tightlist
\item[$\boxtimes$]
  Details
\end{itemize}

The authors report Cronbach's alpha for the threat-based concern measure (\(\alpha\) = .85) and the PANAS (positive affect \(\alpha\) = .86, negative affect \(\alpha\)= .90).

\subsubsection{Do the authors describe any ways they have addressed the validity of their data collection tools/methods?}\label{do-the-authors-describe-any-ways-they-have-addressed-the-validity-of-their-data-collection-toolsmethods}

\begin{itemize}
\tightlist
\item[$\boxtimes$]
  Details
\end{itemize}

The authors use established measures like the PANAS and modular arithmetic task, which have been validated in previous research.

\subsubsection{Was there concealment of study allocation or other key factors from those carrying out measurement of outcome -- if relevant?}\label{was-there-concealment-of-study-allocation-or-other-key-factors-from-those-carrying-out-measurement-of-outcome-if-relevant}

\begin{itemize}
\tightlist
\item[$\boxtimes$]
  No (please specify)
\end{itemize}

The experimenters were likely aware of participants' condition assignments.

\subsubsection{Where were the data collected?}\label{where-were-the-data-collected}

\begin{itemize}
\tightlist
\item[$\boxtimes$]
  Unclear/not stated (please specify)
\end{itemize}

The location of data collection is not specified, but was likely on a university campus.

\subsubsection{Are there other important features of data collection?}\label{are-there-other-important-features-of-data-collection}

\begin{itemize}
\tightlist
\item[$\boxtimes$]
  Details
\end{itemize}

Participants completed tasks in the following order: modular arithmetic learning, power manipulation, stereotype threat manipulation, working memory task, math task, threat-based concern measure.

\subsection{Methods - Data analysis}\label{methods---data-analysis}

\subsubsection{Which methods were used to analyse the data?}\label{which-methods-were-used-to-analyse-the-data}

\begin{itemize}
\tightlist
\item[$\boxtimes$]
  Explicitly stated (please specify)
\end{itemize}

The data were analyzed using ANOVAs and mediational analyses using multiple regression.

\subsubsection{Which statistical methods, if any, were used in the analysis?}\label{which-statistical-methods-if-any-were-used-in-the-analysis}

\begin{itemize}
\tightlist
\item[$\boxtimes$]
  Details
\end{itemize}

ANOVAs, multiple regression, Sobel tests

\subsubsection{What rationale do the authors give for the methods of analysis for the study?}\label{what-rationale-do-the-authors-give-for-the-methods-of-analysis-for-the-study}

\begin{itemize}
\tightlist
\item[$\boxtimes$]
  Details
\end{itemize}

The authors use ANOVAs to examine the effects of power and stereotype threat on the outcome variables. Mediational analyses are used to test whether working memory capacity accounts for the effects on math performance.

\subsubsection{For evaluation studies that use prospective allocation, please specify the basis on which data analysis was carried out.}\label{for-evaluation-studies-that-use-prospective-allocation-please-specify-the-basis-on-which-data-analysis-was-carried-out.}

\begin{itemize}
\tightlist
\item[$\boxtimes$]
  Not applicable (not an evaluation study with prospective allocation)
\end{itemize}

\subsubsection{Do the authors describe any ways they have addressed the reliability of data analysis?}\label{do-the-authors-describe-any-ways-they-have-addressed-the-reliability-of-data-analysis}

\begin{itemize}
\tightlist
\item[$\boxtimes$]
  Details
\end{itemize}

The authors do not explicitly address reliability of data analysis.

\subsubsection{Do the authors describe any ways they have addressed the validity of data analysis?}\label{do-the-authors-describe-any-ways-they-have-addressed-the-validity-of-data-analysis}

\begin{itemize}
\tightlist
\item[$\boxtimes$]
  Details
\end{itemize}

The authors use established analytical techniques appropriate for their design and hypotheses.

\subsubsection{Do the authors describe strategies used in the analysis to control for bias from confounding variables?}\label{do-the-authors-describe-strategies-used-in-the-analysis-to-control-for-bias-from-confounding-variables}

\begin{itemize}
\tightlist
\item[$\boxtimes$]
  Details
\end{itemize}

The authors control for mood and power manipulation check scores in some analyses.

\subsubsection{Please describe any other important features of the analysis.}\label{please-describe-any-other-important-features-of-the-analysis.}

\begin{itemize}
\tightlist
\item[$\boxtimes$]
  Details
\end{itemize}

The authors conducted mediational analyses to examine working memory capacity as a mechanism explaining the effects of power and stereotype threat on math performance.

\subsubsection{Please comment on any other analytic or statistical issues if relevant.}\label{please-comment-on-any-other-analytic-or-statistical-issues-if-relevant.}

\begin{itemize}
\tightlist
\item[$\boxtimes$]
  Details
\end{itemize}

No other major analytic issues are apparent.

\subsection{Results and Conclusions}\label{results-and-conclusions}

\subsubsection{How are the results of the study presented?}\label{how-are-the-results-of-the-study-presented}

\begin{itemize}
\tightlist
\item[$\boxtimes$]
  Details
\end{itemize}

Results are presented using text descriptions, F-statistics, p-values, effect sizes, and figures showing means and standard errors.

\subsubsection{What are the results of the study as reported by authors?}\label{what-are-the-results-of-the-study-as-reported-by-authors}

\begin{itemize}
\tightlist
\item[$\boxtimes$]
  Details
\end{itemize}

The authors found that women in high power conditions did not show reduced working memory capacity or math performance when exposed to stereotype threat, while women in low power and control conditions did. Working memory capacity mediated the effects of power and stereotype threat on math performance.

\subsubsection{Was the precision of the estimate of the intervention or treatment effect reported?}\label{was-the-precision-of-the-estimate-of-the-intervention-or-treatment-effect-reported}

\begin{itemize}
\tightlist
\item
  CONSIDER:

  \begin{itemize}
  \tightlist
  \item
    Were confidence intervals (CIs) reported?
  \end{itemize}
\item[$\boxtimes$]
  No
\end{itemize}

\subsubsection{Are there any obvious shortcomings in the reporting of the data?}\label{are-there-any-obvious-shortcomings-in-the-reporting-of-the-data}

\begin{itemize}
\tightlist
\item[$\boxtimes$]
  No
\end{itemize}

\subsubsection{Do the authors report on all variables they aimed to study as specified in their aims/research questions?}\label{do-the-authors-report-on-all-variables-they-aimed-to-study-as-specified-in-their-aimsresearch-questions}

\begin{itemize}
\tightlist
\item[$\boxtimes$]
  Yes (please specify)
\end{itemize}

The authors report on all key variables: power, stereotype threat, working memory capacity, math performance, and threat-based concern.

\subsubsection{Do the authors state where the full original data are stored?}\label{do-the-authors-state-where-the-full-original-data-are-stored}

\begin{itemize}
\tightlist
\item[$\boxtimes$]
  No
\end{itemize}

\subsubsection{What do the author(s) conclude about the findings of the study?}\label{what-do-the-authors-conclude-about-the-findings-of-the-study}

\begin{itemize}
\tightlist
\item[$\boxtimes$]
  Details
\end{itemize}

The authors conclude that feeling powerful can protect women from the negative effects of stereotype threat on math performance by preserving working memory capacity.

\subsection{Quality of the study - Reporting}\label{quality-of-the-study---reporting}

\subsubsection{Is the context of the study adequately described?}\label{is-the-context-of-the-study-adequately-described}

\begin{itemize}
\tightlist
\item[$\boxtimes$]
  Yes (please specify)
\end{itemize}

The authors provide adequate context, describing relevant research on stereotype threat and power.

\subsubsection{Are the aims of the study clearly reported?}\label{are-the-aims-of-the-study-clearly-reported}

\begin{itemize}
\tightlist
\item[$\boxtimes$]
  Yes (please specify)
\end{itemize}

The aims and hypotheses are clearly stated in the introduction.

\subsubsection{Is there an adequate description of the sample used in the study and how the sample was identified and recruited?}\label{is-there-an-adequate-description-of-the-sample-used-in-the-study-and-how-the-sample-was-identified-and-recruited}

\begin{itemize}
\tightlist
\item[$\boxtimes$]
  No (please specify)
\end{itemize}

Details about sample identification and recruitment are limited.

\subsubsection{Is there an adequate description of the methods used in the study to collect data?}\label{is-there-an-adequate-description-of-the-methods-used-in-the-study-to-collect-data}

\begin{itemize}
\tightlist
\item[$\boxtimes$]
  Yes (please specify)
\end{itemize}

The methods for collecting data on key variables are described in detail.

\subsubsection{Is there an adequate description of the methods of data analysis?}\label{is-there-an-adequate-description-of-the-methods-of-data-analysis}

\begin{itemize}
\tightlist
\item[$\boxtimes$]
  Yes (please specify)
\end{itemize}

The statistical analyses are adequately described.

\subsubsection{Is the study replicable from this report?}\label{is-the-study-replicable-from-this-report}

\begin{itemize}
\tightlist
\item[$\boxtimes$]
  Yes (please specify)
\end{itemize}

The key methods and measures are described in sufficient detail to allow replication.

\subsubsection{Do the authors avoid selective reporting bias?}\label{do-the-authors-avoid-selective-reporting-bias}

\begin{itemize}
\tightlist
\item[$\boxtimes$]
  Yes (please specify)
\end{itemize}

The authors report on all key variables and analyses relevant to their hypotheses.

\subsection{Quality of the study - Methods and data}\label{quality-of-the-study---methods-and-data}

\subsubsection{Are there ethical concerns about the way the study was done?}\label{are-there-ethical-concerns-about-the-way-the-study-was-done}

\begin{itemize}
\tightlist
\item[$\boxtimes$]
  No concerns
\end{itemize}

\subsubsection{Were students and/or parents appropriately involved in the design or conduct of the study?}\label{were-students-andor-parents-appropriately-involved-in-the-design-or-conduct-of-the-study}

\begin{itemize}
\tightlist
\item[$\boxtimes$]
  No (please specify)
\end{itemize}

There is no indication of student or parent involvement in the study design or conduct.

\subsubsection{Is there sufficient justification for why the study was done the way it was?}\label{is-there-sufficient-justification-for-why-the-study-was-done-the-way-it-was}

\begin{itemize}
\tightlist
\item[$\boxtimes$]
  Yes (please specify)
\end{itemize}

The authors provide rationale for their study design based on previous research.

\subsubsection{Was the choice of research design appropriate for addressing the research question(s) posed?}\label{was-the-choice-of-research-design-appropriate-for-addressing-the-research-questions-posed}

\begin{itemize}
\tightlist
\item[$\boxtimes$]
  Yes (please specify)
\end{itemize}

The experimental design was appropriate for testing the causal effects of power and stereotype threat.

\subsubsection{To what extent are the research design and methods employed able to rule out any other sources of error/bias which would lead to alternative explanations for the findings of the study?}\label{to-what-extent-are-the-research-design-and-methods-employed-able-to-rule-out-any-other-sources-of-errorbias-which-would-lead-to-alternative-explanations-for-the-findings-of-the-study}

\begin{itemize}
\tightlist
\item[$\boxtimes$]
  A little (please specify)
\end{itemize}

The experimental design controls for many potential confounds, but some issues like demand characteristics cannot be fully ruled out.

\subsubsection{How generalisable are the study results?}\label{how-generalisable-are-the-study-results}

\begin{itemize}
\tightlist
\item[$\boxtimes$]
  Details
\end{itemize}

The results may generalize to female college students in similar cultural contexts, but generalizability to other populations is unclear.

\subsubsection{Weight of evidence - A: Taking account of all quality assessment issues, can the study findings be trusted in answering the study question(s)?}\label{weight-of-evidence---a-taking-account-of-all-quality-assessment-issues-can-the-study-findings-be-trusted-in-answering-the-study-questions}

\begin{itemize}
\tightlist
\item[$\boxtimes$]
  Medium trustworthiness (please specify)
\end{itemize}

The study uses appropriate methods and analyses, but has some limitations in sampling and generalizability.

\subsubsection{Have sufficient attempts been made to justify the conclusions drawn from the findings so that the conclusions are trustworthy?}\label{have-sufficient-attempts-been-made-to-justify-the-conclusions-drawn-from-the-findings-so-that-the-conclusions-are-trustworthy}

\begin{itemize}
\tightlist
\item[$\boxtimes$]
  High trustworthiness
\end{itemize}

The authors' conclusions are well-supported by their data and analyses.

\section{References}\label{references}

\phantomsection\label{refs}
\begin{CSLReferences}{1}{0}
\bibitem[\citeproctext]{ref-vanlooExperienceFeelingPowerful2013}
Van Loo, K. J., \& Rydell, R. J. (2013). On the experience of feeling powerful: {Perceived} power moderates the effect of stereotype threat on women's math performance. \emph{Personality and Social Psychology Bulletin}, \emph{39}(3), 387--400. \url{https://doi.org/10.1177/0146167212475320}

\end{CSLReferences}


\end{document}
