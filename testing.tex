% Options for packages loaded elsewhere
\PassOptionsToPackage{unicode}{hyperref}
\PassOptionsToPackage{hyphens}{url}
%
\documentclass[
  stu, a4paper]{apa7}
\usepackage{amsmath,amssymb}
\usepackage{iftex}
\ifPDFTeX
  \usepackage[T1]{fontenc}
  \usepackage[utf8]{inputenc}
  \usepackage{textcomp} % provide euro and other symbols
\else % if luatex or xetex
  \usepackage{unicode-math} % this also loads fontspec
  \defaultfontfeatures{Scale=MatchLowercase}
  \defaultfontfeatures[\rmfamily]{Ligatures=TeX,Scale=1}
\fi
\usepackage{lmodern}
\ifPDFTeX\else
  % xetex/luatex font selection
\fi
% Use upquote if available, for straight quotes in verbatim environments
\IfFileExists{upquote.sty}{\usepackage{upquote}}{}
\IfFileExists{microtype.sty}{% use microtype if available
  \usepackage[]{microtype}
  \UseMicrotypeSet[protrusion]{basicmath} % disable protrusion for tt fonts
}{}
\makeatletter
\@ifundefined{KOMAClassName}{% if non-KOMA class
  \IfFileExists{parskip.sty}{%
    \usepackage{parskip}
  }{% else
    \setlength{\parindent}{0pt}
    \setlength{\parskip}{6pt plus 2pt minus 1pt}}
}{% if KOMA class
  \KOMAoptions{parskip=half}}
\makeatother
\usepackage{xcolor}
\usepackage{graphicx}
\makeatletter
\def\maxwidth{\ifdim\Gin@nat@width>\linewidth\linewidth\else\Gin@nat@width\fi}
\def\maxheight{\ifdim\Gin@nat@height>\textheight\textheight\else\Gin@nat@height\fi}
\makeatother
% Scale images if necessary, so that they will not overflow the page
% margins by default, and it is still possible to overwrite the defaults
% using explicit options in \includegraphics[width, height, ...]{}
\setkeys{Gin}{width=\maxwidth,height=\maxheight,keepaspectratio}
% Set default figure placement to htbp
\makeatletter
\def\fps@figure{htbp}
\makeatother
\setlength{\emergencystretch}{3em} % prevent overfull lines
\providecommand{\tightlist}{%
  \setlength{\itemsep}{0pt}\setlength{\parskip}{0pt}}
\setcounter{secnumdepth}{-\maxdimen} % remove section numbering
% Make \paragraph and \subparagraph free-standing
\ifx\paragraph\undefined\else
  \let\oldparagraph\paragraph
  \renewcommand{\paragraph}[1]{\oldparagraph{#1}\mbox{}}
\fi
\ifx\subparagraph\undefined\else
  \let\oldsubparagraph\subparagraph
  \renewcommand{\subparagraph}[1]{\oldsubparagraph{#1}\mbox{}}
\fi
% definitions for citeproc citations
\NewDocumentCommand\citeproctext{}{}
\NewDocumentCommand\citeproc{mm}{%
  \begingroup\def\citeproctext{#2}\cite{#1}\endgroup}
\makeatletter
 % allow citations to break across lines
 \let\@cite@ofmt\@firstofone
 % avoid brackets around text for \cite:
 \def\@biblabel#1{}
 \def\@cite#1#2{{#1\if@tempswa , #2\fi}}
\makeatother
\newlength{\cslhangindent}
\setlength{\cslhangindent}{1.5em}
\newlength{\csllabelwidth}
\setlength{\csllabelwidth}{3em}
\newenvironment{CSLReferences}[2] % #1 hanging-indent, #2 entry-spacing
 {\begin{list}{}{%
  \setlength{\itemindent}{0pt}
  \setlength{\leftmargin}{0pt}
  \setlength{\parsep}{0pt}
  % turn on hanging indent if param 1 is 1
  \ifodd #1
   \setlength{\leftmargin}{\cslhangindent}
   \setlength{\itemindent}{-1\cslhangindent}
  \fi
  % set entry spacing
  \setlength{\itemsep}{#2\baselineskip}}}
 {\end{list}}
\usepackage{calc}
\newcommand{\CSLBlock}[1]{\hfill\break\parbox[t]{\linewidth}{\strut\ignorespaces#1\strut}}
\newcommand{\CSLLeftMargin}[1]{\parbox[t]{\csllabelwidth}{\strut#1\strut}}
\newcommand{\CSLRightInline}[1]{\parbox[t]{\linewidth - \csllabelwidth}{\strut#1\strut}}
\newcommand{\CSLIndent}[1]{\hspace{\cslhangindent}#1}
\ifLuaTeX
\usepackage[bidi=basic]{babel}
\else
\usepackage[bidi=default]{babel}
\fi
\babelprovide[main,import]{english}
% get rid of language-specific shorthands (see #6817):
\let\LanguageShortHands\languageshorthands
\def\languageshorthands#1{}
% Manuscript styling
\usepackage{upgreek}
\captionsetup{font=singlespacing,justification=justified}

% Table formatting
\usepackage{longtable}
\usepackage{lscape}
% \usepackage[counterclockwise]{rotating}   % Landscape page setup for large tables
\usepackage{multirow}		% Table styling
\usepackage{tabularx}		% Control Column width
\usepackage[flushleft]{threeparttable}	% Allows for three part tables with a specified notes section
\usepackage{threeparttablex}            % Lets threeparttable work with longtable

% Create new environments so endfloat can handle them
% \newenvironment{ltable}
%   {\begin{landscape}\centering\begin{threeparttable}}
%   {\end{threeparttable}\end{landscape}}
\newenvironment{lltable}{\begin{landscape}\centering\begin{ThreePartTable}}{\end{ThreePartTable}\end{landscape}}

% Enables adjusting longtable caption width to table width
% Solution found at http://golatex.de/longtable-mit-caption-so-breit-wie-die-tabelle-t15767.html
\makeatletter
\newcommand\LastLTentrywidth{1em}
\newlength\longtablewidth
\setlength{\longtablewidth}{1in}
\newcommand{\getlongtablewidth}{\begingroup \ifcsname LT@\roman{LT@tables}\endcsname \global\longtablewidth=0pt \renewcommand{\LT@entry}[2]{\global\advance\longtablewidth by ##2\relax\gdef\LastLTentrywidth{##2}}\@nameuse{LT@\roman{LT@tables}} \fi \endgroup}

% \setlength{\parindent}{0.5in}
% \setlength{\parskip}{0pt plus 0pt minus 0pt}

% Overwrite redefinition of paragraph and subparagraph by the default LaTeX template
% See https://github.com/crsh/papaja/issues/292
\makeatletter
\renewcommand{\paragraph}{\@startsection{paragraph}{4}{\parindent}%
  {0\baselineskip \@plus 0.2ex \@minus 0.2ex}%
  {-1em}%
  {\normalfont\normalsize\bfseries\itshape\typesectitle}}

\renewcommand{\subparagraph}[1]{\@startsection{subparagraph}{5}{1em}%
  {0\baselineskip \@plus 0.2ex \@minus 0.2ex}%
  {-\z@\relax}%
  {\normalfont\normalsize\itshape\hspace{\parindent}{#1}\textit{\addperi}}{\relax}}
\makeatother

\makeatletter
\usepackage{etoolbox}
\patchcmd{\maketitle}
  {\section{\normalfont\normalsize\abstractname}}
  {\section*{\normalfont\normalsize\abstractname}}
  {}{\typeout{Failed to patch abstract.}}
\patchcmd{\maketitle}
  {\section{\protect\normalfont{\@title}}}
  {\section*{\protect\normalfont{\@title}}}
  {}{\typeout{Failed to patch title.}}
\makeatother

\usepackage{xpatch}
\makeatletter
\xapptocmd\appendix
  {\xapptocmd\section
    {\addcontentsline{toc}{section}{\appendixname\ifoneappendix\else~\theappendix\fi\\: #1}}
    {}{\InnerPatchFailed}%
  }
{}{\PatchFailed}
\keywords{keywords\newline\indent Word count: X}
\usepackage{lineno}

\linenumbers
\usepackage{csquotes}
\makeatletter
\renewcommand{\paragraph}{\@startsection{paragraph}{4}{\parindent}%
  {0\baselineskip \@plus 0.2ex \@minus 0.2ex}%
  {-1em}%
  {\normalfont\normalsize\bfseries\typesectitle}}

\renewcommand{\subparagraph}[1]{\@startsection{subparagraph}{5}{1em}%
  {0\baselineskip \@plus 0.2ex \@minus 0.2ex}%
  {-\z@\relax}%
  {\normalfont\normalsize\bfseries\itshape\hspace{\parindent}{#1}\textit{\addperi}}{\relax}}
\makeatother

\usepackage{pdflscape}
\ifLuaTeX
  \usepackage{selnolig}  % disable illegal ligatures
\fi
\usepackage{bookmark}
\IfFileExists{xurl.sty}{\usepackage{xurl}}{} % add URL line breaks if available
\urlstyle{same}
\hypersetup{
  pdftitle={The title},
  pdfauthor={First Author1},
  pdflang={en-EN},
  pdfkeywords={keywords},
  hidelinks,
  pdfcreator={LaTeX via pandoc}}

\title{The title}
\author{First Author\textsuperscript{1}}
\date{}


\shorttitle{Title}

\authornote{

Enter author note here.

The authors made the following contributions. First Author: Conceptualization.

Correspondence concerning this article should be addressed to First Author, Postal address. E-mail: \href{mailto:my@email.com}{\nolinkurl{my@email.com}}

}

\affiliation{\vspace{0.5cm}\textsuperscript{1} Wilhelm-Wundt-University}

\begin{document}
\maketitle

\begin{lltable}

\begin{TableNotes}[para]
\normalsize{\textit{Note.} This table summarizes studies investigating neural activation patterns under stereotype threat. The 'Variables' column focuses on brain areas and networks of interest, such as the amygdala, prefrontal cortex, default mode network, and salience network. 'Methods of Data Analysis' includes neuroimaging techniques like fMRI and EEG. 'Results' highlight changes in neural activation patterns related to stereotype threat.}
\end{TableNotes}

\begin{longtable}{m{1.5cm}m{3cm}m{2.5cm}m{3cm}m{3cm}m{3cm}m{3.5cm}m{1.5cm}}\noalign{\getlongtablewidth\global\LTcapwidth=\longtablewidth}
\caption{\label{tab:unnamed-chunk-1}Overview of the Included Papers for Hypothesis 1}\\
\toprule
Citation & Study Design & Population & Research Questions & Variables & Methods of Data Analysis & Results & Hypothesis confirmed\\
\midrule
\endfirsthead
\caption*{\normalfont{Table \ref{tab:unnamed-chunk-1} continued}}\\
\toprule
Citation & Study Design & Population & Research Questions & Variables & Methods of Data Analysis & Results & Hypothesis confirmed\\
\midrule
\endhead
Beilock et al. (2007) & Experimental & Female college students in US & Behavioral tasks & Stereotype threat, working memory efficiency & ANOVA & Reduced performance on high-demand problems under threat & Yes\\
Dunst et al. (2013) & Experimental & 58 secondary school students in Austria & EEG & Stereotype threat, neural efficiency, task performance & ANOVA & Higher cortical activation under threat & Partially\\
Forbes et al. (2015) & Experimental & 58 participants (25 White, 33 minorities) & EEG & DMN phase-locking, error estimates, self-doubt & Regression models & DMN phase-locking may mitigate stereotype threat effects & Yes\\
Forbes et al. (2008) & Experimental & 57 minority undergraduates & EEG & ERN, Pe, task performance & Repeated measures analysis & Smaller ERN amplitudes under threat & Partially\\
Jończyk et al. (2022) & Experimental & 23 female undergraduates in US & EEG & Creativity, alpha power & Repeated measures ANOVA & Increased alpha power after threat & Partially\\
Krendl et al. (2008) & Experimental & 28 female undergraduates & fMRI & Neural activation, math performance & Mixed-model ANOVA & Increased vACC activation, decreased cognitive region activation under threat & Partially\\
Mangels et al. (2012) & Prospective & 68 participants & EEG & Math performance, ERP responses, learning success & ANOVA & LPP and learning success link more pronounced under threat & Partially\\
Wu and Zhao (2021) & Experimental & 48 female undergraduates in China & RS-fMRI & RSDC of brain regions & Mixed-effect analysis & Increased RSDC in DMN areas, decreased in cerebellum and hippocampus & Partially\\
\bottomrule
\addlinespace
\insertTableNotes
\end{longtable}

\end{lltable}

\begin{lltable}

\begin{TableNotes}[para]
\normalsize{\textit{Note.} This table presents studies examining cognitive control processes under stereotype threat. The 'Variables' column includes both cognitive processes (e.g., inhibition, updating, shifting) and related performance measures. 'Methods of Data Analysis' specifies cognitive tasks used, such as the Stroop task, n-back task, or task-switching paradigms. 'Results' emphasize changes in cognitive control performance under stereotype threat conditions.}
\end{TableNotes}

\begin{longtable}{m{1.5cm}m{3cm}m{2.5cm}m{3cm}m{3cm}m{3cm}m{3.5cm}m{1.5cm}}\noalign{\getlongtablewidth\global\LTcapwidth=\longtablewidth}
\caption{\label{tab:unnamed-chunk-2}Overview of the Included Papers for Hypothesis 2}\\
\toprule
Citation & Study Design & Population & Research Questions & Variables & Methods of Data Analysis & Results & Hypothesis confirmed\\
\midrule
\endfirsthead
\caption*{\normalfont{Table \ref{tab:unnamed-chunk-2} continued}}\\
\toprule
Citation & Study Design & Population & Research Questions & Variables & Methods of Data Analysis & Results & Hypothesis confirmed\\
\midrule
\endhead
Guardabassi and Tomasetto (2020) & Cross-sectional & 176 primary school children & N-back task & BMI, stereotype threat, working memory & Mixed-effects models & zBMI negatively correlated with working memory under threat & Partially\\
Hirnstein et al. (2014) & Factorial & 136 participants (66 male, 70 female) & Cognitive tests & Stereotype threat, sex, group composition, cognitive performance & ANOVA & Performance decreased on 4W and perceptual speed under threat & Weakly\\
Jordano and Touron (2017) & Experimental & 120 female undergraduates & OSPAN task, mind-wandering probes & Stereotype threat, mind-wandering, task performance & ANOVA & Increased mind-wandering, decreased math performance under threat & Partially\\
Krendl et al. (2008) & Experimental & 28 female undergraduates & fMRI & Neural activation, math performance & Mixed-model ANOVA & Increased vACC activation, decreased cognitive region activation under threat & Yes\\
Lin et al. (2023) & Cross-sectional & 153 female undergraduates & Spatial perspective-taking, executive function tests & Stereotype threat, executive function, spatial performance & ANCOVA, mediation analysis & Decreased performance, impaired inhibition and updating under threat & Partially\\
Rydell et al. (2014) & Experimental & 340 undergraduates across 3 experiments & Executive function tasks, math tests & Stereotype threat, executive function, math performance & ANOVA, mediation analysis & Impaired inhibition and updating, decreased math performance under threat & Mostly\\
Ståhl et al. (2012) & Experimental & 335 students across 3 experiments & Stroop task, math task & Stereotype threat, regulatory focus, cognitive control & ANOVA & Initial increase then decrease in cognitive control under threat (prevention focus) & Mostly\\
Wister et al. (2013) & Experimental & 92 female undergraduates & Stroop test, SAT-like math test & Menstruation threat, cognitive performance & MANOVA & Impaired Stroop performance under menstruation threat & Partially\\
Wulandari and Hendrawan (2020) & Experimental & 168 undergraduates (91 female) & Letter fluency test & Stereotype threat activation, gender, task difficulty & ANOVA & No significant effects of threat on performance & No\\
\bottomrule
\addlinespace
\insertTableNotes
\end{longtable}

\end{lltable}

\begin{lltable}

\begin{TableNotes}[para]
\normalsize{\textit{Note.} This table outlines studies investigating working memory impairment under stereotype threat. The 'Variables' column focuses on working memory measures and associated performance indicators. 'Methods of Data Analysis' details specific working memory tasks employed, such as complex span tasks, operational span tasks, or reading span tests. 'Results' highlight changes in working memory capacity and performance under stereotype threat.}
\end{TableNotes}

\begin{longtable}{m{1.5cm}m{3cm}m{2.5cm}m{3cm}m{3cm}m{3cm}m{3.5cm}m{1.5cm}}\noalign{\getlongtablewidth\global\LTcapwidth=\longtablewidth}
\caption{\label{tab:unnamed-chunk-3}Overview of the Included Papers for Hypothesis 3}\\
\toprule
Citation & Study Design & Population & Research Questions & Variables & Methods of Data Analysis & Results & Hypothesis confirmed\\
\midrule
\endfirsthead
\caption*{\normalfont{Table \ref{tab:unnamed-chunk-3} continued}}\\
\toprule
Citation & Study Design & Population & Research Questions & Variables & Methods of Data Analysis & Results & Hypothesis confirmed\\
\midrule
\endhead
Bedyńska et al. (2020) & Cross-sectional & 319 male secondary school students & Working memory tasks & Chronic stereotype threat, working memory, language achievement & Mediation analysis & Stereotype threat negatively impacted working memory capacity & Yes\\
Bedyńska et al. (2018) & Cross-sectional & 624 female secondary school students & Working memory tasks & Chronic stereotype threat, working memory, math achievement & Mediation analysis & Working memory mediated stereotype threat and math achievement & Yes\\
Beilock et al. (2007) & Experimental & Female college students in US & Modular Arithmetic task & Stereotype threat, working memory efficiency & ANOVA & Reduced performance on high-demand problems under threat & Yes\\
Brown and Harkins (2016) & Experimental & 73 female undergraduates & SART, math test & Stereotype threat, SART framing, mind-wandering & ANOVA & Support for mere effort account, not working memory impairment & No\\
Guardabassi and Tomasetto (2020) & Cross-sectional & 176 primary school children & N-back task & BMI, stereotype threat, working memory & Mixed-effects models & zBMI negatively correlated with working memory under threat & Yes\\
Hutchison et al. (2013) & Experimental & 187 men & Stroop task, OSPAN & Working memory capacity, stereotype threat, Stroop performance & Regression analysis & Stroop effect larger under threat for low WMC individuals & Partially\\
Jamieson and Harkins (2007) & Experimental & 224 undergraduates across 4 experiments & Saccade tasks, N-back task & Stereotype threat, task type, cognitive load & ANOVA & Support for mere effort account in most conditions & Mostly No\\
Johns et al. (2008) & Experimental & 176 participants across 3 experiments & Working memory task, math test & Stereotype threat, emotion regulation, working memory & ANOVA, mediation analysis & Working memory impaired under threat, mediated math performance & Yes\\
Pennington et al. (2019) & Experimental & 124 female university students & Anti-saccade task, math task & Stereotype condition, task performance & ANOVA & No significant effects of threat on performance & No\\
Rydell et al. (2009) & Experimental & 57 female undergraduates & Vowel counting task, math problems & Gender stereotype, college student stereotype, working memory & ANOVA, mediation analysis & Working memory capacity mediated stereotype effects on math performance & Yes\\
Schmader et al. (2009) & Experimental & 188 participants across 2 experiments & Reading Span Test & Stereotype threat, anxiety, working memory & Regression analysis & Anxiety predicted lower working memory under stereotype threat & Partially\\
Schmader and Johns (2003) & Experimental & 159 undergraduates across 3 experiments & OSPAN, math test & Stereotype threat, working memory capacity, math performance & ANCOVA, mediation analysis & Working memory capacity reduced under threat, mediated math performance & Yes\\
Tine and Gotlieb (2013) & Experimental & 71 undergraduates & Math test, working memory test & Multiple stereotype threats, math and working memory performance & ANOVA & Working memory impaired under various stereotype threats & Yes\\
Van Loo and Rydell (2013) & Experimental & 131 female undergraduates & Letter-memory task, math test & Power prime, stereotype threat, working memory & ANOVA, mediation analysis & High power prime protected working memory from stereotype threat effects & Mostly\\
\bottomrule
\addlinespace
\insertTableNotes
\end{longtable}

\end{lltable}

\section{Methods}\label{methods}

We report how we determined our sample size, all data exclusions (if any), all manipulations, and all measures in the study.

\subsection{Participants}\label{participants}

\subsection{Material}\label{material}

\begin{table}[tbp]

\begin{center}
\begin{threeparttable}

\caption{\label{tab:unnamed-chunk-4}}

\small{

\begin{tabular}{ll}
\toprule
Hypothesis & \multicolumn{1}{c}{Query}\\
\midrule
Hypothesis 1 & SELECT * FROM papers WHERE hypothesis = 1\\
\bottomrule
\end{tabular}

}

\end{threeparttable}
\end{center}

\end{table}

\section{HELLO WORLD}\label{hello-world}

This is the anova result: \(F(2, 27) = 4.85\), \(\mathit{MSE} = 0.39\), \(p = .016\).

\subsection{Procedure}\label{procedure}

\subsection{Data analysis}\label{data-analysis}

We used R (Version 4.4.1; R Core Team, 2024) and the R-packages \emph{citr} (Version 0.3.2; Aust, 2019), \emph{kableExtra} (Version 1.4.0; Zhu, 2024), \emph{papaja} (Version 0.1.2.9000; Aust \& Barth, 2023), \emph{RefManageR} (Version 1.4.0; McLean, 2017), \emph{rmarkdown} (Version 2.27; Xie et al., 2018, 2020), and \emph{tinylabels} (Version 0.2.4; Barth, 2023) for all our analyses.

\section{Results}\label{results}

\section{Discussion}\label{discussion}

\newpage

\section{References}\label{references}

\phantomsection\label{refs}
\begin{CSLReferences}{1}{0}
\bibitem[\citeproctext]{ref-R-citr}
Aust, F. (2019). \emph{Citr: 'RStudio' add-in to insert markdown citations}. \url{https://github.com/crsh/citr}

\bibitem[\citeproctext]{ref-R-papaja}
Aust, F., \& Barth, M. (2023). \emph{{papaja}: {Prepare} reproducible {APA} journal articles with {R Markdown}}. \url{https://github.com/crsh/papaja}

\bibitem[\citeproctext]{ref-R-tinylabels}
Barth, M. (2023). \emph{{tinylabels}: Lightweight variable labels}. \url{https://cran.r-project.org/package=tinylabels}

\bibitem[\citeproctext]{ref-R-RefManageR}
McLean, M. W. (2017). RefManageR: Import and manage BibTeX and BibLaTeX references in r. \emph{The Journal of Open Source Software}. \url{https://doi.org/10.21105/joss.00338}

\bibitem[\citeproctext]{ref-R-base}
R Core Team. (2024). \emph{R: A language and environment for statistical computing}. R Foundation for Statistical Computing. \url{https://www.R-project.org/}

\bibitem[\citeproctext]{ref-R-rmarkdown_a}
Xie, Y., Allaire, J. J., \& Grolemund, G. (2018). \emph{R markdown: The definitive guide}. Chapman; Hall/CRC. \url{https://bookdown.org/yihui/rmarkdown}

\bibitem[\citeproctext]{ref-R-rmarkdown_b}
Xie, Y., Dervieux, C., \& Riederer, E. (2020). \emph{R markdown cookbook}. Chapman; Hall/CRC. \url{https://bookdown.org/yihui/rmarkdown-cookbook}

\bibitem[\citeproctext]{ref-R-kableExtra}
Zhu, H. (2024). \emph{kableExtra: Construct complex table with 'kable' and pipe syntax}. \url{https://CRAN.R-project.org/package=kableExtra}

\end{CSLReferences}


\end{document}
