% Options for packages loaded elsewhere
\PassOptionsToPackage{unicode}{hyperref}
\PassOptionsToPackage{hyphens}{url}
%
\documentclass[
  stu, a4paper]{apa7}
\usepackage{amsmath,amssymb}
\usepackage{iftex}
\ifPDFTeX
  \usepackage[T1]{fontenc}
  \usepackage[utf8]{inputenc}
  \usepackage{textcomp} % provide euro and other symbols
\else % if luatex or xetex
  \usepackage{unicode-math} % this also loads fontspec
  \defaultfontfeatures{Scale=MatchLowercase}
  \defaultfontfeatures[\rmfamily]{Ligatures=TeX,Scale=1}
\fi
\usepackage{lmodern}
\ifPDFTeX\else
  % xetex/luatex font selection
\fi
% Use upquote if available, for straight quotes in verbatim environments
\IfFileExists{upquote.sty}{\usepackage{upquote}}{}
\IfFileExists{microtype.sty}{% use microtype if available
  \usepackage[]{microtype}
  \UseMicrotypeSet[protrusion]{basicmath} % disable protrusion for tt fonts
}{}
\makeatletter
\@ifundefined{KOMAClassName}{% if non-KOMA class
  \IfFileExists{parskip.sty}{%
    \usepackage{parskip}
  }{% else
    \setlength{\parindent}{0pt}
    \setlength{\parskip}{6pt plus 2pt minus 1pt}}
}{% if KOMA class
  \KOMAoptions{parskip=half}}
\makeatother
\usepackage{xcolor}
\usepackage{graphicx}
\makeatletter
\def\maxwidth{\ifdim\Gin@nat@width>\linewidth\linewidth\else\Gin@nat@width\fi}
\def\maxheight{\ifdim\Gin@nat@height>\textheight\textheight\else\Gin@nat@height\fi}
\makeatother
% Scale images if necessary, so that they will not overflow the page
% margins by default, and it is still possible to overwrite the defaults
% using explicit options in \includegraphics[width, height, ...]{}
\setkeys{Gin}{width=\maxwidth,height=\maxheight,keepaspectratio}
% Set default figure placement to htbp
\makeatletter
\def\fps@figure{htbp}
\makeatother
\setlength{\emergencystretch}{3em} % prevent overfull lines
\providecommand{\tightlist}{%
  \setlength{\itemsep}{0pt}\setlength{\parskip}{0pt}}
\setcounter{secnumdepth}{-\maxdimen} % remove section numbering
% Make \paragraph and \subparagraph free-standing
\ifx\paragraph\undefined\else
  \let\oldparagraph\paragraph
  \renewcommand{\paragraph}[1]{\oldparagraph{#1}\mbox{}}
\fi
\ifx\subparagraph\undefined\else
  \let\oldsubparagraph\subparagraph
  \renewcommand{\subparagraph}[1]{\oldsubparagraph{#1}\mbox{}}
\fi
\ifLuaTeX
\usepackage[bidi=basic]{babel}
\else
\usepackage[bidi=default]{babel}
\fi
\babelprovide[main,import]{english}
% get rid of language-specific shorthands (see #6817):
\let\LanguageShortHands\languageshorthands
\def\languageshorthands#1{}
% Manuscript styling
\usepackage{upgreek}
\captionsetup{font=singlespacing,justification=justified}

% Table formatting
\usepackage{longtable}
\usepackage{lscape}
% \usepackage[counterclockwise]{rotating}   % Landscape page setup for large tables
\usepackage{multirow}		% Table styling
\usepackage{tabularx}		% Control Column width
\usepackage[flushleft]{threeparttable}	% Allows for three part tables with a specified notes section
\usepackage{threeparttablex}            % Lets threeparttable work with longtable

% Create new environments so endfloat can handle them
% \newenvironment{ltable}
%   {\begin{landscape}\centering\begin{threeparttable}}
%   {\end{threeparttable}\end{landscape}}
\newenvironment{lltable}{\begin{landscape}\centering\begin{ThreePartTable}}{\end{ThreePartTable}\end{landscape}}

% Enables adjusting longtable caption width to table width
% Solution found at http://golatex.de/longtable-mit-caption-so-breit-wie-die-tabelle-t15767.html
\makeatletter
\newcommand\LastLTentrywidth{1em}
\newlength\longtablewidth
\setlength{\longtablewidth}{1in}
\newcommand{\getlongtablewidth}{\begingroup \ifcsname LT@\roman{LT@tables}\endcsname \global\longtablewidth=0pt \renewcommand{\LT@entry}[2]{\global\advance\longtablewidth by ##2\relax\gdef\LastLTentrywidth{##2}}\@nameuse{LT@\roman{LT@tables}} \fi \endgroup}

% \setlength{\parindent}{0.5in}
% \setlength{\parskip}{0pt plus 0pt minus 0pt}

% Overwrite redefinition of paragraph and subparagraph by the default LaTeX template
% See https://github.com/crsh/papaja/issues/292
\makeatletter
\renewcommand{\paragraph}{\@startsection{paragraph}{4}{\parindent}%
  {0\baselineskip \@plus 0.2ex \@minus 0.2ex}%
  {-1em}%
  {\normalfont\normalsize\bfseries\itshape\typesectitle}}

\renewcommand{\subparagraph}[1]{\@startsection{subparagraph}{5}{1em}%
  {0\baselineskip \@plus 0.2ex \@minus 0.2ex}%
  {-\z@\relax}%
  {\normalfont\normalsize\itshape\hspace{\parindent}{#1}\textit{\addperi}}{\relax}}
\makeatother

\makeatletter
\usepackage{etoolbox}
\patchcmd{\maketitle}
  {\section{\normalfont\normalsize\abstractname}}
  {\section*{\normalfont\normalsize\abstractname}}
  {}{\typeout{Failed to patch abstract.}}
\patchcmd{\maketitle}
  {\section{\protect\normalfont{\@title}}}
  {\section*{\protect\normalfont{\@title}}}
  {}{\typeout{Failed to patch title.}}
\makeatother

\usepackage{xpatch}
\makeatletter
\xapptocmd\appendix
  {\xapptocmd\section
    {\addcontentsline{toc}{section}{\appendixname\ifoneappendix\else~\theappendix\fi\\: #1}}
    {}{\InnerPatchFailed}%
  }
{}{\PatchFailed}
\keywords{keywords\newline\indent Word count: X}
\usepackage{lineno}

\linenumbers
\usepackage{csquotes}
\makeatletter
\renewcommand{\paragraph}{\@startsection{paragraph}{4}{\parindent}%
  {0\baselineskip \@plus 0.2ex \@minus 0.2ex}%
  {-1em}%
  {\normalfont\normalsize\bfseries\typesectitle}}

\renewcommand{\subparagraph}[1]{\@startsection{subparagraph}{5}{1em}%
  {0\baselineskip \@plus 0.2ex \@minus 0.2ex}%
  {-\z@\relax}%
  {\normalfont\normalsize\bfseries\itshape\hspace{\parindent}{#1}\textit{\addperi}}{\relax}}
\makeatother

\usepackage{pdflscape}
\usepackage{amsmath}
\usepackage{setspace}
\AtBeginEnvironment{tabular}{\singlespacing}
\AtBeginEnvironment{lltable}{\singlespacing}
\AtBeginEnvironment{tablenotes}{\doublespacing}
\captionsetup[table]{font={stretch=1.5}}
\captionsetup[figure]{font={stretch=1.5}}
\ifLuaTeX
  \usepackage{selnolig}  % disable illegal ligatures
\fi
\usepackage{bookmark}
\IfFileExists{xurl.sty}{\usepackage{xurl}}{} % add URL line breaks if available
\urlstyle{same}
\hypersetup{
  pdftitle={The title},
  pdfauthor={First Author1},
  pdflang={en-EN},
  pdfkeywords={keywords},
  hidelinks,
  pdfcreator={LaTeX via pandoc}}

\title{The title}
\author{First Author\textsuperscript{1}}
\date{}


\shorttitle{Title}

\authornote{

Enter author note here.

The authors made the following contributions. First Author: Conceptualization.

Correspondence concerning this article should be addressed to First Author, Postal address. E-mail: \href{mailto:my@email.com}{\nolinkurl{my@email.com}}

}

\affiliation{\vspace{0.5cm}\textsuperscript{1} Wilhelm-Wundt-University}

\begin{document}
\maketitle

\begin{verbatim}
## Warning in matrix(value, n, p): data length [12] is not a sub-multiple or
## multiple of the number of columns [8]
\end{verbatim}

\begin{lltable}

\begin{TableNotes}[para]
\normalsize{\textit{Note.} This table summarises studies investigating working memory impairment under stereotype threat. The 'Variables' column focuses on working memory measures and associated performance indicators. 'Methods of Data Analysis' details specific working memory tasks employed, such as complex span tasks, operational span tasks, or reading span tests. 'Results' highlight changes in working memory capacity and performance under stereotype threat. Asterisks indicate the significance level: *p < .05, **p < .01, ***p < .001.}
\end{TableNotes}

\begin{longtable}{m{1.5cm}m{3cm}m{2.5cm}m{3cm}m{3cm}m{3cm}m{3.5cm}m{1.5cm}}\noalign{\getlongtablewidth\global\LTcapwidth=\longtablewidth}
\caption{\label{tab:h3_table}Overview of the Included Papers for Hypothesis 3}\\
\toprule
Citation & Study Design & Population & Research Questions & Variables & Methods of Data Analysis & Results & Hypothesis confirmed\\
\midrule
\endfirsthead
\caption*{\normalfont{Table \ref{tab:h3_table} continued}}\\
\toprule
Citation & Study Design & Population & Research Questions & Variables & Methods of Data Analysis & Results & Hypothesis confirmed\\
\midrule
\endhead
Bedyńska et al. (2020) & Cross-sectional & 319 male secondary school students & Effects of chronic stereotype threat on working memory and language achievement; counting span task, set switching task, and spatial location memory task (capacity) & IV: Chronic stereotype threat, Gender identification; DV: Working memory, Language achievement & Mediation analysis & Stereotype threat negatively impacted working memory capacity, with the latter mediating the relationship between stereotype threat and language achievement. $b$ = 2.81, $\beta$ = 0.45, $SE$ = 0.06, $p$ < .001, 95\% CI [[0.34, 0.55]]. Higher gender identification moderated the effect of stereotype threat on working memory. $r$ = 0.32. & Yes\\
Bedyńska et al. (2018) & Cross-sectional & 624 female secondary school students & Effects of chronic stereotype threat on working memory and maths achievement; Functional Aspects of Working Memory Test (capacity, accuracy) & IV: Chronic stereotype threat, Gender identification; DV: Working memory, Maths achievement & Mediation analysis & Working memory mediated the relationship between stereotype threat and maths achievement. $\beta$ = 0.50, indirect effect $\beta$ = -0.14, 95\% CI [[-0.20, -0.07]]. Higher gender identification moderated the negative effect of stereotype threat on working memory. $b$ = -0.01, $\beta$ = -0.29, $SE$ = 0.14, $p$ = .039. $r$ = 0.20. & Yes\\
Beilock et al. (2007) & Experimental & Experiment 1: 31 female college students; Experiment 3A: 33 female college students; Experiment 4: 30 female college students; Experiment 5: 33 female college students & Effects of stereotype threat on working memory and its influence on unrelated tasks; modular arithmetic (processing speed); n-back task (capacity, accuracy) & IV: Group (stereotype threat vs. control), Problem working memory demand (low vs. high), Block (baseline vs. posttest); DV: Accuracy, Reaction time & ANOVA & High-demand problems showed a significant decrease in accuracy at the post-test, CI [81.00\% - 97.00\%]; $\textit{d}$ = 0.61. $\textit{F}$(1,29) = 11.18, $\eta^{2}_\text{p}$ = 0.28. & Yes\\
Brown and Harkins (2016) & Experimental & 73 female undergraduates & Effects of stereotype threat on mind-wandering and task performance; SART (processing speed, accuracy) & IV: Condition (stereotype threat vs. control), SART framing (related vs. unrelated); DV: Mind-wandering (SART performance) & ANOVA & Significant effect of the mere effort account: commission errors $\textit{F}$(1, 69) = 28.78, $p$ < .001, $\eta^{2}_\text{p}$ = 0.29. Counter-hypothesis not supported. & No\\
Guardabassi and Tomasetto (2020) & Cross-sectional & 176 primary school children & Effects of BMI and stereotype threat on working memory; N-back task (capacity, accuracy) & IV: BMI, Stereotype threat; DV: Working memory & Mixed-effects models & zBMI negatively correlated with working memory under threat. $F$ = & 12.40\\
Hutchison et al. (2013) & Experimental & 187 men & Effects of stereotype threat on working memory and Stroop performance; OSPAN (capacity, accuracy) & IV: Working memory capacity, List congruency, Stereotype threat condition; DV: Stroop task performance & Regression analysis & Stroop effect larger under threat for low WMC individuals. $\beta$ = 0.12, $\beta$ = -0.11, $\beta$ = 0.24*. & Partially\\
Jamieson and Harkins (2007) & Experimental & 224 undergraduates across 4 experiments & Effects of stereotype threat on task performance requiring inhibitory control; saccade tasks (processing speed, accuracy) & IV: Condition (stereotype threat vs. control), Task type (antisaccade vs. prosaccade), Cognitive load; DV: Accuracy, Reaction time, Eye movements & ANOVA & Support for mere effort account in most conditions. Anti-saccade task: $F$(1, 72) = 17.28, $p$ < .001, $d$ = 0.98. Condition x Task: $F$(1, 72) = 4.85, $p$ = .050. & Mostly No\\
Johns et al. (2008) & Experimental & 176 participants across 3 experiments & Effects of stereotype threat on working memory and emotion regulation; reading span task (capacity, accuracy) & IV: Condition (stereotype threat vs. control), Emotion regulation strategy; DV: Working memory capacity, Maths performance, Self-reported anxiety & ANOVA, mediation analysis & Working memory impaired under threat, mediated maths performance. $t$(55) = 2.31, $\beta$ = 0.30*. & Yes\\
Pennington et al. (2019) & Experimental & 124 female university students & Effects of stereotype condition on task performance; anti-saccade task (accuracy, processing speed) & IV: Stereotype condition; DV: Task performance & ANOVA & No significant effects of threat on performance. Anti-saccade task: $F$(2, 58) = 0.30, $p$ = .750, $\eta^{2}_\text{p}$ = 0.01. & No\\
Rydell et al. (2009) & Experimental & 57 female undergraduates & Effects of multiple social identities on stereotype threat and working memory; vowel counting task (capacity, accuracy) & IV: Gender stereotype, College student stereotype; DV: Working memory capacity, Maths performance & ANOVA, mediation analysis & Working memory capacity mediated stereotype effects on maths performance. $F$(1, 53) = 6.01, $p$ = .020, $\eta^{2}_\text{p}$ = 0.10. Sobel test: $z$ = 1.96, $p$ = .050. & Yes\\
Schmader et al. (2009) & Experimental & 188 participants across 2 experiments & Effects of stereotype threat on anxiety and working memory; Reading Span Test (capacity, processing speed) & IV: Prime condition, Self-reported anxiety; DV: Working memory performance & Regression analysis & Anxiety predicted lower working memory under stereotype threat. $\beta$ = -0.20, $p$ = .050. Prime x Anxiety interaction significant: $\beta$ = -0.30, $p$ < .040. & Partially\\
Schmader and Johns (2003) & Experimental & 151 undergraduates across 3 experiments & Effects of stereotype threat on working memory and maths performance; reading span task (capacity, accuracy) & IV: Condition (stereotype threat vs. control); DV: Working memory capacity, Maths test performance & ANOVA, mediation analysis & Working memory capacity predicted maths performance. $F$(1, 54) = 23.84, $p$ < .001. Mediation: Sobel test $z$ = 2.26, $p$ < .020. & Yes\\
Tine and Gotlieb (2013) & Experimental & 71 undergraduates & Effects of gender-, race-, and income-based stereotype threat on working memory; Automated Working Memory Assessment (capacity, accuracy) & IV: Gender, Race, Income level, Number of stigmatized aspects; DV: Working memory performance & ANOVA & Significant effects of stereotype threat on working memory performance. $F$(1, 68) = 4.91, $p$ < .050, $\eta^{2}_\text{p}$ = 0.07; $F$(1, 68) = 16.73, $p$ < .001, $\eta^{2}_\text{p}$ = 0.20. & Yes\\
Van Loo and Rydell (2013) & Experimental & 131 female undergraduates & Effects of power prime on stereotype threat and working memory; letter-memory task (capacity, accuracy) & IV: Power prime, Stereotype threat condition; DV: Working memory capacity, Maths performance & ANOVA, mediation analysis & High power prime protected working memory from stereotype threat effects. $F$(2, 125) = 13.38***, mediated by working memory capacity. $z$ = -3.53***. & Mostly\\
\bottomrule
\addlinespace
\insertTableNotes
\end{longtable}

\end{lltable}


\end{document}
